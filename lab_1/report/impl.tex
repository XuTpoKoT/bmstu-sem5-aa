\chapter{Технологическая часть}

В данном разделе приведены требования к ПО, средства реализации и листинги кода.

\section{Требования к ПО}

Используемое ПО должно предоставлять возможность измерения процессорного времени.

\section{Средства реализации}

Для реализации данной лабораторной работы был выбран язык программирования Golang \cite{golang} и среда разработки Goland, которая позволяет замерять процессорное время с помощью пакета \texttt{СGo} \cite{cgo}.

\clearpage
\section{Листинг кода}

В листингах \ref{lst:LMatr}--\ref{lst:DLMatr} приведены реализации алгоритмов нахождения расстояния Левенштейна и Дамерау-Левенштейна.

\begin{lstlisting}[label=lst:LMatr,caption=Функция нахождения расстояния Левенштейна с заполнением матрицы (начало)]
func LevenshteinMatrix(str1, str2 string) int {
	n := len(str1)
	m := len(str2)
	if n == 0 {
		return m
	} else if m == 0 {
		return n
	}
	
	b1 := make([]int, n+1)
	b2 := make([]int, n+1)
	
	for i := 0; i < n+1; i++ {
		b2[i] = i
	}
\end{lstlisting}

\clearpage
\begin{lstlisting}[caption=Функция нахождения расстояния Левенштейна с заполнением матрицы (окончание)]
	for i := 1; i < m+1; i++ {
		swap(&b1, &b2)
		
		b2[0] = i
		for j := 1; j < n+1; j++ {
			flag := 1
			if str1[j-1] == str2[i-1] {
				flag = 0
			}
			
			res := min(b2[j-1]+1, b1[j]+1, b1[j-1]+flag)
			
			b2[j] = res
		}		
	}
	
	return b2[n]
}
	
\end{lstlisting}

\begin{lstlisting}[label=lst:DLRec,caption=Функция нахождения расстояния Дамерау-Левенштейна без использованиея рекурсии (начало)]
func DamerauLevenshteinMatrix(str1, str2 string) int {
	n := len(str1)
	m := len(str2)
	
	if n == 0 {
		return m
	} else if m == 0 {
		return n
	}
\end{lstlisting}

\clearpage
\begin{lstlisting}[label=lst:DLRec,caption=Функция нахождения расстояния Дамерау-Левенштейна без использованиея рекурсии (окончание)]	
	b1 := make([]int, n+1)
	b2 := make([]int, n+1)
	b3 := make([]int, n+1)
	
	for i := 0; i < n+1; i++ {
		b3[i] = i
	}
	
	for i := 1; i < m+1; i++ {
		swapLeft(&b1, &b2, &b3)
		
		b3[0] = i
		for j := 1; j < n+1; j++ {
			flag := 1
			if str1[j-1] == str2[i-1] {
				flag = 0
			}
			
			res := min(b3[j-1]+1, b2[j]+1, b2[j-1]+flag)
			
			if i > 1 && j > 1 && str1[j-1] == str2[i-2] && str1[j-2] == str2[i-1] {
				res = min(res, b1[j-2] + 1)
			}
			
			b3[j] = res
		}
	}
	
	return b3[n]
}
	
\end{lstlisting}

\begin{lstlisting}[label=lst:DLRecCache,caption=Функция нахождения расстояния Дамерау-Левенштейна с использованием рекурсии.]
func DamerauLevenshteinRec(str1, str2 string) int {
	n := len(str1)
	m := len(str2)
	
	if n == 0 {
		return m
	} else if m == 0 {
		return n
	}
	
	flag := 1
	if str1[n-1] == str2[m-1] {
		flag = 0
	}
	
	res := min(DamerauLevenshteinRec(str1[:n-1], str2)+1,
	DamerauLevenshteinRec(str1, str2[:m-1])+1,
	DamerauLevenshteinRec(str1[:n-1], str2[:m-1])+flag)
	
	if n >= 2 && m >= 2 && str1[n-1] == str2[m-2] && str1[n-2] == str2[m-1] {
		cur := DamerauLevenshteinRec(str1[:n-2], str2[:m-2]) + 1
		if cur < res {
			res = cur
		}
	}
	
	return res
}
	
\end{lstlisting}

\begin{lstlisting}[label=lst:DLMatr,caption=Функция нахождения расстояния Дамерау-Левенштейна с использованием рекурсии с кешем (начало)]
func damerauLevenshteinRecCache(str1, str2 string, cache [][]int) int {
	n := len(str1)
	m := len(str2)
	
	if n == 0 {
		return m
	} else if m == 0 {
		return n
	}
	
	if cache[n-1][m-1] != -1 {
		return cache[n-1][m-1]
	}
	
	res := damerauLevenshteinRecCache(str1[:n-1], str2, cache) + 1
	
	cur := damerauLevenshteinRecCache(str1, str2[:m-1], cache) + 1
	if cur < res {
		res = cur
	}

	flag := 1
	if str1[n-1] == str2[m-1] {
		flag = 0
	}
	
	cur = damerauLevenshteinRecCache(str1[:n-1], str2[:m-1], cache) + flag
\end{lstlisting}

\clearpage
\begin{lstlisting}[caption=Функция нахождения расстояния Дамерау-Левенштейна с использованием рекурсии с кешем (окончание)]		
	if cur < res {
		res = cur
	}
	
	if n >= 2 && m >= 2 && str1[n-1] == str2[m-2] && str1[n-2] == str2[m-1] {
		cur = damerauLevenshteinRecCache(str1[:n-2], str2[:m-2], cache) + 1
		if cur < res {
			res = cur
		}
	}
	
	cache[n-1][m-1] = res
	return res
}
	
\end{lstlisting}

\clearpage
\section{Функциональные тесты}
В таблице \ref{tabular:functional_test} приведены функциональные тесты для алгоритмов вычисления расстояния Левенштейна (в таблице столбец подписан "Левенштейн") и Дамерау-Левенштейна (в таблице --- "Дамерау-Л."). Все тесты пройдены успешно.


\begin{table}[h]
	\begin{center}
		\caption{\label{tabular:functional_test} Функциональные тесты}
		\begin{tabular}{|c|c|c|c|c|}
			\hline
			& \multicolumn{2}{c|}{Входные данные} & \multicolumn{2}{c|}{Ожидаемый результат} \\
			\hline
			№&Строка 1&Строка 2&Левенштейн&Дамерау-Л. \\
			\hline
			1&cat&cute&2&2 \\
			\hline
			2&cute&cat&2&2 \\
			\hline
			3&dog&dog&0&0 \\
			\hline
			4&toook&t&4&4 \\
			\hline
			5&пустая строка&пустая строка&0&0 \\
			\hline
			8&пустая строка&33&2&2 \\
			\hline
			9&1234&пустая строка&4&4 \\
			\hline
			10&sample&samlpee&2&2 \\
			\hline
			11&abcde&abced&2&1 \\
			\hline
			12&abcde&based&4&2 \\
			\hline
		\end{tabular}
	\end{center}
\end{table}
