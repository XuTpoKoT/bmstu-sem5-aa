\chapter{Аналитическая часть}
В этом разделе будут представлены описания алгоритмов нахождения расстояний Левенштейна и Дамерау-Левенштейна и их практическое применение.
\section{Матричный алгоритм нахождения \newline расстояния Левенштейна}

\textbf{Расстояние Левенштейна} между двумя строками --- это минимальное количество операций вставки, удаления и замены, необходимых для превращения одной строки в другую.
При этом каждая операция имеет свою цену (штраф).

Рассмотрим матрицу $A$ размером $(length(s_{1})+ 1) \cdot ((length(s_{2}) + 1)$, где $length(S)$ --- длина строки $S$. Пусть значение в ячейке $[i, j]$ матрицы равно расстоянию между префиксом $s_{1}$ длины $i$ и префиксом $s_{2}$ длины $j$. У элементов первой строки значение равно индексу столбца, у элементов первого столбца --- индексу строки.

Остальные ячейки заполняем в соответствии с формулой (\ref{eq:Lmat}).
\begin{equation}
	\label{eq:Lmat}
	A[i][j] = min \begin{cases}
		A[i-1][j] + 1;\\
		A[i][j-1] + 1;\\
		A[i-1][j-1] + m(s_{1}[i], s_{2}[j]).
	\end{cases}
\end{equation}

Функция $m$ определена как:
\begin{equation}
	\label{eq:Lm}
	m(s_{1}[i], s_{2}[j]) = \begin{cases}
		0, &\text{если $s_{1}[i-1] = s_{2}[j-1]$;}\\
		1, &\text{иначе.}
	\end{cases}
\end{equation}

В результате расстоянием Левенштейна будет ячейка матрицы с индексами $i = length(s_{1}$) и $j = length(s_{2})$.

\section{Рекурсивный алгоритм нахождения расстояния Дамерау-Левенштейна}

\textbf{Расстояние Дамерау-Левенштейна} между двумя строками --- это минимальное количество операций вставки, удаления, замены и транспозиции (перестановки двух соседних символов), необходимых для перевода одной строки в другую. Является модификацией расстояния Левенштейна.

Рекурсивный алгоритм считает редакционное расстояние для двух строк $s_{1}$ и $s_{2}$ по рекуррентной формуле (\ref{eq:DRec}), где $i$ --- длина префикса $s_{1}$, $j$ --- длина префикса $s_{2}$:
\begin{equation}
	\label{eq:DRec}
	d(i, j) = \begin{cases}
		\max(i, j), \text{если }\min(i, j) = 0;\\
		\min \lbrace \\
			\qquad d(i, j-1) + 1;\\
			\qquad d(i-1, j) + 1;\\
			\qquad d(i-1, j-1) + m(a[i], b[j]);\\
			\qquad \begin{cases} 
				d(i-2, j-2) + 1, &\text{если }i,j > 1,\\
				\qquad &s_{1}[i] = s_{2}[j-1],\\
				\qquad &s_{2}[j] = s_{1}[i-1];\\
				\qquad \infty, & \text{иначе.}
			\end{cases}\\
		\rbrace, \text{иначе.}
		\end{cases}
\end{equation}

Функция $f$ определена как:
\begin{equation}
	\label{eq:Lm}
	m(i, j) = \begin{cases}
		0, &\text{если $s_{1}[i-1] = s_{2}[j-1]$;}\\
		1, &\text{иначе.}
	\end{cases}
\end{equation}

\section{Рекурсивный алгоритм нахождения расстояния Дамерау-Левенштейна с использованием кеша}

Рекурсивный алгоритм можно оптимизировать, если записывать найденные промежуточные расстояния в кеш-матрицу. Перед началом рассчёта требуется инициализировать ячейки матрицы значением $-1$. При рекурсивном вызове требуется проверить, было ли значение вычислено ранее, --- проверить, находится ли в соответствующей ячейке матрицы $-1$.
Таким образом, на поиск уже найденных расстояний время не тратится --- их значения берутся из матрицы.

\section{Матричный алгоритм нахождения \newline расстояния Дамерау-Левенштейна}

При больших $i, j$ прямая реализация формулы (\ref{eq:DRec}) может быть неэффективна по времени, так как некоторые значения $ D(i, j)$ вычисляются несколько раз. Для оптимизации алгоритма можно хранить промежуточные значения в матрице размером $(length(s_{1})+ 1) \times \times((length(s_{2}) + 1)$, где $length(S)$ --- длина строки $S$. Значение в ячейке $[i, j]$ матрицы A равно расстоянию между префиксом $s_{1}$ длины $i$ и префиксом $s_{2}$ длины $j$. У элементов первой строки значение равно индексу столбца, у элементов первого столбца --- индексу строки.

Остальные ячейки заполняем в соответствии с формулой (\ref{eq:DMat}).
\begin{equation}
	\label{eq:DMat}
	A[i][j] = min \begin{cases}
		A[i-1][j] + 1;\\
		A[i][j-1] + 1;\\
		A[i-1][j-1] + m(s_{1}[i], s_{2}[j]);\\
		\begin{cases} 
			A[i-2][j-2] + 1, &\text{если }i,j > 1,\\
			\qquad &s_{1}[i] = s_{2}[j-1],\\
			\qquad &s_{2}[j] = s_{1}[i-1];\\
			\qquad \infty, & \text{иначе.}
		\end{cases}\\
	\end{cases}
\end{equation}

Функция $m$ определена как
\begin{equation}
	\label{eq:m2}
	m(s_{1}[i], s_{2}[j]) = \begin{cases}
		0, &\text{если $s_{1}[i - 1] = s_{2}[j - 1]$;}\\
		1, &\text{иначе.}
	\end{cases}
\end{equation}

В результате расстоянием Дамерау-Левенштейна будет ячейка матрицы с индексами $i = length(s_{1}$) и $j = length(s_{2})$.




