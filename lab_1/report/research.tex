\chapter{Исследовательская часть}

В данном разделе произведено сравнение алгоритмов.

\section{Технические характеристики}

Технические характеристики устройства, на котором выполнялись замеры времени:

\begin{itemize}
	\item операционная система --- Ubuntu 22.04.1 Linux x86\_64;
	\item оперативная память --- 8 ГБ;
	\item процессор --- AMD Ryzen 5 3550H \cite{amd}.
\end{itemize}

Замеры проводились на ноутбуке, включенном в сеть электропитания. Во время замеров ноутбук не был нагружен сторонними приложениями.

\section{Время выполнения алгоритмов}

На рисунке \ref{img:g1} представлен график, иллюстрирующий зависимость времени работы от длины строк для матричного алгоритма поиска расстояния Левенштейна и матричного алгоритма поиска расстояния Дамерау-Левенштейна.

\begin{figure}[h!]
	\centering
	\begin{tikzpicture}
	\begin{axis}[	
		height = 0.4\paperheight, 
		width = 0.65\paperwidth,
		legend pos = north west,
		table/col sep=comma,
		xlabel={длина строк},
		ylabel={время, нс},
		]
		\legend{ 
			Левенштейн матричный, 
			Д.-Левенштейн матричный,
			Д.-Левенштейн рекурсивный с кешем,
		};
		\addplot [
			solid, 
			draw = blue,
			mark = *, 
			mark options = {
				scale = 1.5, 
				fill = blue, 
				draw = black
			}
		] table [x={size}, y={time}] {f1.csv};
		\addplot [
			dashed, 
			draw = red,
			mark = 10-pointed star, 
			mark options = {
				scale = 1.5, 
				draw = red
			}
		] table [x={size}, y={time}] {f2.csv};
		\addplot [
			densely dotted, 
			draw = green,
			mark = *, 
			mark options = {
				scale = 1.5, 
				fill = green, 
				draw = black
			}
		] table [x={size}, y={time}] {f5.csv};
		
	\end{axis}
	\end{tikzpicture}
	\caption{Сравнение матричного алгоритма поиска расстояния Левенштейна и матричного алгоритма поиска расстояния Дамерау-Левенштейна}
	\label{img:g1}
\end{figure}

\clearpage
На рисунке \ref{img:g2} представлен график, иллюстрирующий зависимость времени работы от длины строк для рекурсивных алгоритмов поиска расстояния Дамерау-Левенштейна с использованием кеша и без.

\begin{figure}[h!]
	\centering
	\begin{tikzpicture}
	\begin{axis}[	
		height = 0.4\paperheight, 
		width = 0.65\paperwidth,
		legend pos = north west,
		table/col sep=comma,
		xlabel={длина строк},
		ylabel={время, нс},
		]
		\legend{ 
			Рекурсивный Д.-Лев. без кеша, 
			Рекурсивный Д.-Лев. с кешем, 
		};
		\addplot [
		solid, 
		draw = blue,
		mark = *, 
		mark options = {
			scale = 1.5, 
			fill = blue, 
			draw = black
		}
		] table [x={size}, y={time}] {f3.csv};
		\addplot [
		dashed, 
		draw = red,
		mark = star, 
		mark options = {
			scale = 1.5, 
			draw = red
		}
		] table [x={size}, y={time}] {f4.csv};
		
	\end{axis}
	\end{tikzpicture}
	\caption{Сравнение рекурсивных алгоритмов поиска расстояния Дамерау-Левенштейна с использованием кеша и без}
	\label{img:g2}
\end{figure}
 
\section{Использование памяти}

Пусть длина строки $S_1$ --- $n$, длина строки $S_2$ --- $m$. Максимальная глубина стека вызовов при рекурсивной реализации равна сумме длин входящий строк.

Обозначим: char --- тип, используемый для хранения символа строки, int --- тип, используемый для хранения чисел.

Рассчитаем затраты по памяти для матричных алгоритмов поиска расстояния Левенштейна и Дамерау-Левенштейна:
\begin{itemize}
	\item строки $S_1$, $S_2$ --- $(m + n) \cdot sizeof(char)$;
	\item матрица --- $(m + 1) \cdot (n + 1) \cdot sizeof(int)$;
	\item длины строк --- $2 \cdot sizeof(int)$;
	\item вспомогательные переменные --- $3 \cdot sizeof(int)$;
	\item итого --- $(m + n) \cdot sizeof(char) + (m + 1) \cdot (n + 1) \cdot sizeof(int) + 5 \cdot sizeof(int)$.
\end{itemize}

Рассчитаем затраты по памяти для рекурсивного алгоритма поиска расстояния Дамерау-Левенштейна(для каждого вызова):
\begin{itemize}
	\item строки $S_1$, $S_2$ --- $(m + n) \cdot sizeof(char)$;
	\item длины строк --- $2 \cdot sizeof(int)$;
	\item вспомогательные переменные --- $3 \cdot sizeof(int)$;
	\item адрес возврата --- $8$ байт;
	\item итого --- $(m + n) \cdot sizeof(char) + 5 \cdot sizeof(int) + 8$ байт.
\end{itemize}

Высота дерева рекурсивных вызовов $max(m, n) + 1$. Тогда максимальная глубина стека равна 
\begin{equation}
	\begin{gathered}
		M_{rec} = (min(m, n) + 1) \cdot ((m + n) \cdot sizeof(char) + \\
		+ 5 \cdot sizeof(int) + 8).
	\end{gathered}
\end{equation}

Рассчитаем затраты по памяти для рекурсивного алгоритма поиска расстояния Дамерау-Левенштейна с использованием кеша (для каждого вызова): 
\begin{itemize}
	\item строки $S_1$, $S_2$ --- $(m + n) \cdot sizeof(char)$;
	\item длины строк --- $2 \cdot sizeof(int)$;
	\item вспомогательные переменные ---  $3 \cdot sizeof(int)$;
	\item ссылка на матрицу --- $8$ байт;
	\item адрес возврата --- $8$ байт;
	\item итого --- $(m + n) \cdot sizeof(char) + 5 \cdot sizeof(int) + 16$ байт.
\end{itemize}

Память для хранения матрицы(для всех вызовов общая)
\begin{equation}
	M_{matr} = (m + 1) \cdot (n + 1) \cdot sizeof(int).
\end{equation}

Максимальная глубина стека равна
\begin{equation}
	\begin{gathered}
	M_{cash} = (min(m, n) + 1) \cdot ((m + n) \cdot sizeof(char) + \\
	+ 5 \cdot sizeof(int) + 16) + (m + 1) \cdot (n + 1) \cdot sizeof(int).
	\end{gathered}
\end{equation}

\section*{Вывод}

Алгоритм нахождения расстояния Дамерау-Левенштейна по времени выполнения незначительно отличается от алгоритма нахождения расстояния Левенштейна (для слов длиной 100 символов 89 мкс против 80 мкс).

По расходу памяти итеративный алгоритм проигрывают рекурсивному: максимальный размер используемой памяти в них растёт как произведение длин строк, в то время как у рекурсивного алгоритма --- как сумма длин строк.

Рекурсивный алгоритм с заполнением матрицы превосходит по времени работы простой рекурсивный (для слов длиной 10 символов 16 мкс против 43353 мкс) и сравним с матричным алгоритмом.


