\chapter*{Заключение}
\addcontentsline{toc}{chapter}{Заключение}

Цель достигнута. В ходе выполнения лабораторной работы были решены следующие задачи:

\begin{itemize}
    \item изучены и реализованы алгоритмы нахождения расстояний Левенштейна и Дамерау-Левенштейна;
	\item выполнена теоретическая оценка затрат алгоритмов по памяти;
	\item выполнена экспериментальная оценка затрат алгоритмов по времени;
	\item проведено сравнение алгоритмов по проведённым оценкам.
\end{itemize}

Алгоритм нахождения расстояния Дамерау-Левенштейна по производительности схож с алгоритмом нахождения расстояния Левенштейна (для слов длиной 100 быстрее на 9\%).

Рекурсивный алгоритм с заполнением матрицы эффективнее по времени работы, чем простой рекурсивный (для слов длиной 10 в 2700 раз быстрее) и незначительно отличается от матричной реализации (для слов длиной 100 в 4 раз медленнее). Однако по расходу памяти рекурсивные алгоритмы эффективнее матричных, так как максимальный размер используемой памяти в них растёт как произведение длин строк, в то время как у рекурсивного алгоритма --- как сумма длин строк.


