\chapter*{Введение}
\addcontentsline{toc}{chapter}{Введение}

Целью данной лабораторной работы является получение практических навыков динамического программирования на примере реализации алгоритмов Левенштейна и Дамерау-Левенштейна.

\textbf{Расстояние Левенштейна}  (редакционное расстояние, дистанция редактирования) --- метрика, измеряющая разность между двумя последовательностями символов. Она определяется как минимальное количество односимвольных операций (вставки, удаления, замены), необходимых для превращения одной строки в другую \cite{Levenshtein}. 

Расстояние Левенштейна и его обобщения применяются:
\begin{itemize}
	\item для исправления ошибок в слове (в поисковых системах, базах данных, при вводе текста, при автоматическом распознавании отсканированного текста или речи);
	\item для сравнения текстовых файлов утилитой diff и ей подобными (здесь роль «символов» играют строки, а роль «строк» --- файлы);
	\item в биоинформатике \cite{Bioinf}. 
\end{itemize}

\textbf{Расстояние Дамерау-Левенштейна} (названо в честь учёных Фредерика Дамерау и Владимира Левенштейна) --- это мера разницы двух строк символов, определяемая как минимальное количество операций вставки, удаления, замены и транспозиции (перестановки двух соседних символов), необходимых для перевода одной строки в другую. Является модификацией расстояния Левенштейна, так как к операциям вставки, удаления и замены символов, определённых в расстоянии Левенштейна добавлена операция транспозиции (перестановки) символов \cite{DamLevenshtein}. 

Задачами данной лабораторной являются:
\begin{itemize}
	\item изучение и реализация алгоритмов Левенштейна и Дамерау-Левенштейна нахождения редакционного расстояния между строками;
	\item выполнение теоретической либо экспериментальной оценки затрат алгоритмов по памяти;
	\item выполнение экспериментальной оценки затрат алгоритмов по времени;
	\item сравнение алгоритмов по проведённым оценкам.
\end{itemize}
