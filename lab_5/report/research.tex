\chapter{Исследовательская часть}

В данном разделе произведено сравнение однопоточной и многопоточной версии алгоритма.

\section{Технические характеристики}

Технические характеристики устройства, на котором выполнялись замеры времени:

\begin{itemize}
	\item операционная система --- Ubuntu 22.04.1 Linux x86\_64;
	\item оперативная память --- 8 ГБ;
	\item процессор --- AMD Ryzen 5 3550H \cite{amd}.
\end{itemize}

Замеры проводились на ноутбуке, включенном в сеть электропитания. Во время замеров ноутбук не был нагружен сторонними приложениями.

\section{Время выполнения алгоритма}
На рисунке \ref{img:g1} представлен график, иллюстрирующий зависимость времени работы однопоточной и многопоточной версии алгоритма от количества заявок.

\clearpage
\begin{figure}[h!]
	\centering
	\begin{tikzpicture}
		\begin{loglogaxis}[	
			height = 0.4\paperheight, 
			width = 0.65\paperwidth,
			legend pos = north west,
			table/col sep=comma,
			xlabel={кол-во заявок},
			ylabel={время, мкс},
			]
			\legend{ 
				Конвейер, 
				Посл. обработка, 
			};
			\addplot [
			solid, 
			draw = blue,
			mark = *, 
			mark options = {
				scale = 1.5, 
				fill = blue, 
				draw = black
			}
			] table [x={size}, y={time}] {parallel.csv};
			\addplot [
			dotted, 
			draw = red,
			mark = star, 
			mark options = {
				scale = 1.5, 
				draw = red
			}
			] table [x={size}, y={time}] {consistently.csv};		
		\end{loglogaxis}
	\end{tikzpicture}
	\caption{Сравнение реализаций алгоритма}
	\label{img:g1}
\end{figure}

\section*{Вывод} 
При количестве заявок 1000 конвейерная обработка превосходит последовательную примерно в 2 раза.

