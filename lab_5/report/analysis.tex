\chapter{Аналитическая часть}
В этом разделе представлено описание алгоритма.

\section{Сжатие Хаффмана}

Одним из основных методов, используемыми для сжатия данных в базах данных в настоящее время является кодирование Хаффмана на основе кодового бинарного дерева.
Код Хаффмана является неравномерным и префиксным.
Неравномерность кода означает, что символы имеют разную длину кодового слова (размер кода), коды символов, которые чаще встречаются в тексте, имеют меньший размер, а коды редко встречающихся символов, имеют больший размер. 
Префиксный называется такая кодировка, в которой ни один код не является началом другого кода, таким образом выполняется условие Фано и достигается однозначность при декодировании \cite{huf}.

Метод сжатия информации на основе двоичных кодирующих деревьев был предложен Д. А. Хаффманом в 1952.
Идея алгоритма состоит в том, чтобы наиболее часто встречающиеся символы имели более короткие коды, символы, встречающиеся реже всего, имели очень длинный код.
Алгоритм сжатия данных Хаффмана, как канонический, так и его адаптивные версии, обладают достаточно высокой эффективностью и лежат в основе многих других методов, используемых в алгоритмах сжатия данных.
Каждому символу в кодируемом тексте присваивается вес, равный частоте его появления.
Затем формируется бинарное дерево кодировки, в котором каждый символ является листом.
Путь вправо --- добавление единицы в текущий код, путь влево --- добавление нуля \cite{huf}.

Алгоритм сжатия можно описать с помощью 3 этапов.
\begin{enumerate}
	\item Создание конечного узла для каждого символа и добавление его в очередь приоритетов.
	\item Пока в очереди больше одного узла: удаление из очереди двух узлов с наивысшим приоритетом (самой низкой частотой); создание нового внутреннего узла с этими двумя узлами в качестве дочерних элементов и частотой, равной сумме частот обоих узлов; добавление нового узла в очередь приоритетов.
	\item Обход сформированного дерева и формирование кода Хаффмана.
\end{enumerate}

Итак, строка aabacdab будет закодирована в 00110100011011 \\(0|0|11|0|100|011|0|11), с помощью следующих кодов:
\begin{itemize}
	\item a --- 0;
	\item b --- 11;
	\item c --- 100;
	\item d --- 011.
\end{itemize}