\chapter{Технологическая часть}

В данном разделе представлены требования к программному обеспечению, средства реализации и листинги кода.

\section{Требования к программному \newline обеспечению}

Используемое программное обеспечение должно предоставлять возможность измерения процессорного времени и работы с потоками.

\section{Средства реализации}

Для реализации данной лабораторной работы был выбран язык программирования С++ \cite{c} и среда разработки CLion, которая позволяет замерять процессорное время с помощью пакета \texttt{<ctime>} \cite{ctime} и работать с потоками, используя класс std::thread \cite{thread}.

\section{Реализация алгоритма}

В листингах \ref{lst:Request} и \ref{lst:Conveyor1}--\ref{lst:Conveyor3} приведены реализации классов заявки и конвейера, в листингах \ref{lst:stage1}--\ref{lst:encode} приведены раелизации этапов алгоритма Хаффмана.

\clearpage
\begin{lstlisting}[label=lst:Request,caption=Класс Request]
class Request {
public:
	Request(string srcData) : srcData{srcData} {}	
	string srcData;
	string result;
	unordered_map<char, int> freq;
	priority_queue<shared_ptr<Node>, vector<shared_ptr<Node>>, NodeComparator> pq;
	unordered_map<char, string> huffmanCode;
	
	static shared_ptr<Request> createRequest() {
		return make_shared<Request>(StringGenerator::genString(10));
	}
};
\end{lstlisting}

\begin{lstlisting}[label=lst:Conveyor1,caption=Класс Conveyor (часть 1)]
class Conveyor {
public:
	Conveyor(){}
	
	void runParallel(size_t cntRequests) {
		for (size_t i = 0; i < cntRequests; i++) {
			auto r = Request::createRequest();
			q1.push(r);
		}
		
		this->threads[0] = std::thread(&Conveyor::stage1, this);
		this->threads[1] = std::thread(&Conveyor::stage2, this);
		this->threads[2] = std::thread(&Conveyor::stage3, this);
		
		for (int i = 0; i < THRD_CNT; i++) {
			this->threads[i].join();
		}
		
		for (size_t i = 0; i < cntRequests; i++) {
			auto r = processedRequests[i];
			Huffman::printInfo(r);
		}
	}
\end{lstlisting}

\begin{lstlisting}[label=lst:Conveyor2,caption=Класс Conveyor (часть 2)]
private:
	void stage1() {
		while (!this->q1.empty()) {
			auto r = q1.front();
			Huffman::stage1(r);
			
			q2.push(r);
			q1.pop();
		}	
	}
	
	void stage2() {
		do {
			if (!this->q2.empty()) {
				auto r = q2.front();
				Huffman::stage2(r);
				
				q3.push(r);
				q2.pop();
			}
		} while(!q1.empty() || !q2.empty());
	}
	
	void stage3() {
		do {
			if (!this->q3.empty()) {
				auto r = q3.front();
				Huffman::stage3(r);
				
				processedRequests.push_back(r);
				q3.pop();
			}
		} while(!q1.empty() || !q2.empty() || !q3.empty());
	}
	
private:
	std::thread threads[THRD_CNT];
	std::vector<std::shared_ptr<Request>> processedRequests;

\end{lstlisting}
\clearpage
\begin{lstlisting}[label=lst:Conveyor3,caption=Класс Conveyor (часть 3)]
	std::queue<std::shared_ptr<Request>> q1;
	std::queue<std::shared_ptr<Request>> q2;
	std::queue<std::shared_ptr<Request>> q3;
};
\end{lstlisting}

\begin{lstlisting}[label=lst:stage1,caption=1-й этап алгоритма Хаффмана]
void stage1(shared_ptr<Request> r) {
	for (char ch: r->srcData) {
		r->freq[ch]++;
	}
	
	for (auto pair: r->freq) {
		r->pq.push(make_shared<Node>(pair.first, pair.second, nullptr, nullptr));
	}
}
\end{lstlisting}

\begin{lstlisting}[label=lst:stage2,caption=2-й этап алгоритма Хаффмана]
void stage2(shared_ptr<Request> r) {
	while (r->pq.size() > 1) {
		shared_ptr<Node> left = r->pq.top(); r->pq.pop();
		shared_ptr<Node> right = r->pq.top(); r->pq.pop();
		
		int sum = left->freq + right->freq;
		r->pq.push(make_shared<Node>('\0', sum, left, right));
	}
}
\end{lstlisting}

\begin{lstlisting}[label=lst:stage3,caption=3-й этап алгоритма Хаффмана]
void stage3(shared_ptr<Request> r) {
	encode(r->pq.top(), "", r->huffmanCode);
	
	for (char ch: r->srcData) {
		r->result += r->huffmanCode[ch];
	}
}
\end{lstlisting}
\clearpage
\begin{lstlisting}[label=lst:encode,caption=Формирование кодов Хаффмана]
void encode(shared_ptr<Node> root, string str, unordered_map<char, string> &huffmanCode) {
	if (root == nullptr)
		return;
	
	if (!root->left && !root->right) {
		huffmanCode[root->ch] = str;
	}
	
	encode(root->left, str + "0", huffmanCode);
	encode(root->right, str + "1", huffmanCode);
}
\end{lstlisting}
