\chapter{Технологическая часть}

В данном разделе приведены требования к ПО, средства реализации и листинги кода.

\section{Требования к ПО}

Используемое ПО должно предоставлять возможность измерения процессорного времени.

\section{Средства реализации}

Для реализации данной лабораторной работы был выбран язык программирования С++ \cite{c} и среда разработки CLion, которая позволяет замерять процессорное время с помощью пакета \texttt{<ctime>} \cite{ctime}.

\section{Листинг кода}

В листингах \ref{lst:St}--\ref{lst:Cols},  приведены реализации алгоритмов.

\clearpage
\begin{lstlisting}[label=lst:St,caption=Функция умножения матриц по стандартному алгоритму]
static void mulStand(const Matrix& m1, const Matrix& m2, Matrix& res) {
	for (int i = 0; i < m1.rows; i++) {
		for (int j = 0; j < m1.cols; j++)
		for (int k = 0; k < m2.rows; k++)
		res.data[i][j] += m1.data[i][k] * m2.data[k][j];
	}
}	
\end{lstlisting}

\begin{lstlisting}[label=lst:VinSlow,caption=Функция умножения матриц по алгоритму Винограда (начало)]
static void mulVinograd0(const Matrix& A, const Matrix& B, Matrix& res) {
	std::vector<int> mulH = A.preprocRows0();
	std::vector<int> mulV = B.preprocCols0();
	for (int i = 0; i < A.rows; i++) {
		for (int j = 0; j < B.cols; j++) {
			res.data[i][j] = -(mulH[i] + mulV[j]);
			for (int k = 0; k < A.cols/2; k++) {
				res.data[i][j] = res.data[i][j] +
				(A.data[i][k*2]+B.data[k*2+1][j])*
				(A.data[i][k*2+1]+B.data[k*2][j]);
			}
		}
	}
\end{lstlisting}	

\clearpage
\begin{lstlisting}[label=lst:VinSlow2,caption=Функция умножения матриц по алгоритму Винограда (окончание)]
	if (A.cols%2 != 0) {
		for (int i = 0; i < A.rows; i++) {
			for (int j = 0; j < B.cols; j++) {
				res.data[i][j] = res.data[i][j] + A.data[i][A.cols-1]*B.data[A.cols-1][j];
			}
		}
	}
}	
\end{lstlisting}

\begin{lstlisting}[label=lst:RowsSlow,caption=Функция предобработки строк]
std::vector<int> preprocRows0() const {
	std::vector<int> res(rows);
	for (int i = 0; i < rows; i++) {
		for (int j = 0; j < cols/2; j++) {
			res[i] = res[i] + data[i][j*2] * data[i][j*2+1];
		}
	}
	return res;
}	
\end{lstlisting}

\clearpage
\begin{lstlisting}[label=lst:ColsSlow,caption=Функция предобработки столбцов]
std::vector<int> preprocCols0() const {
	std::vector<int> res(cols);
	for (int i = 0; i < cols; i++) {
		for (int j = 0; j < rows/2; j++) {
			res[i] = res[i] + data[j*2][i] * data[j*2+1][i];
		}
	}
	return res;
}
\end{lstlisting}

\begin{lstlisting}[label=lst:Vin,caption=Функция умножения матриц по алгоритму Винограда с оптимизациями (начало)]
static void mulVinograd3(const Matrix& A, const Matrix& B, Matrix& res) {
	std::vector<int> mulH = A.preprocRows3();
	std::vector<int> mulV = B.preprocCols3();
	for (int i = 0; i < A.rows; i++) {
		for (int j = 0; j < B.cols; j++) {
			int tmp = mulH[i] + mulV[j];
			for (int k = 1; k < A.cols; k += 2) {
				tmp += (A.data[i][k-1] + B.data[k][j]) *
				(A.data[i][k] + B.data[k-1][j]);
			}
			res.data[i][j] = tmp;
		}
	}
\end{lstlisting}
\clearpage
\begin{lstlisting}[label=lst:Vin,caption=Функция умножения матриц по алгоритму Винограда с оптимизациями (окончание)]
	if ((A.cols&1) != 0) {
		int n1 = A.cols - 1;
		for (int i = 0; i < A.rows; i++) {
			for (int j = 0; j < B.cols; j++) {
				res.data[i][j] += A.data[i][n1] * B.data[n1][j];
			}
		}
	}
}
\end{lstlisting}

\begin{lstlisting}[label=lst:Rows,caption=Функция предобработки строк с оптимизациями]
std::vector<int> preprocRows3() const {
	std::vector<int> res(rows);
	for (int i = 0; i < rows; i++) {
		for (int j = 1; j < cols; j += 2) {
			res[i] -= data[i][j-1] * data[i][j];
		}
	}
	return res;
}
\end{lstlisting}

\clearpage
\begin{lstlisting}[label=lst:Cols,caption=Функция предобработки столбцов с оптимизациями]
std::vector<int> preprocCols3() const {
	std::vector<int> res(cols);
	for (int i = 0; i < cols; i++) {
		for (int j = 1; j < rows; j += 2) {
			res[i] -= data[j-1][i] * data[j][i];
		}
	}
	return res;
}
\end{lstlisting}

\section{Функциональные тесты}
В таблице~\ref{tabular:test_rec} приведены тесты для функций, реализующих стандартный алгоритм умножения матриц, алгоритм Винограда и оптимизированный алгоритм Винограда. Все тесты пройдены успешно.

\begin{table}[h!]
	\caption{\label{tabular:test_rec} Тестирование функций}
	\begin{center}
		\begin{tabular}{c@{\hspace{7mm}}c@{\hspace{7mm}}c@{\hspace{7mm}}}
			\hline
			Матрица 1 & Матрица 2 &Ожидаемый результат \\ 
			\hline
			\vspace{4mm}
			$\begin{pmatrix}
				6
			\end{pmatrix}$ &
			$\begin{pmatrix}
				7
			\end{pmatrix}$ &
			$\begin{pmatrix}
				42
			\end{pmatrix}$ \\
			\vspace{2mm}
			$\begin{pmatrix}
				1 & 2 & 3
			\end{pmatrix}$ &
			$\begin{pmatrix}
				3 \\
				2 \\
				1
			\end{pmatrix}$ &
			$\begin{pmatrix}
				10
			\end{pmatrix}$ \\
			\vspace{4mm}
			$\begin{pmatrix}
				1\\
				2\\
				3
			\end{pmatrix}$ &
			$\begin{pmatrix}
				3 & 2 & 1
			\end{pmatrix}$ &
			$\begin{pmatrix}
				3 & 2 & 1\\
				6 & 4 & 2 \\
				9 & 6 & 3
			\end{pmatrix}$ \\
			\vspace{4mm}
			$\begin{pmatrix}
				4 & 4 & 4\\
				2 & 2 & 2 \\
				3 & 3 & 3
			\end{pmatrix}$ &
			$\begin{pmatrix}
				1\\
				1\\
				1
			\end{pmatrix}$ &
			$\begin{pmatrix}
				12\\
				6\\
				9
			\end{pmatrix}$ \\
			\vspace{4mm}
			$\begin{pmatrix}
				5 & 0\\
				0 & 4
			\end{pmatrix}$ &
			$\begin{pmatrix}
				0 & 7\\
				8 & 0
			\end{pmatrix}$ &
			$\begin{pmatrix}
				0 & 35\\
				32 & 0
			\end{pmatrix}$ \\		
			\vspace{4mm}
			$\begin{pmatrix}
				2 & 3
			\end{pmatrix}$ &
			$\begin{pmatrix}
				2 & 3
			\end{pmatrix}$ &
			Неверный размер\\
		\end{tabular}
	\end{center}
	
\end{table}

