\chapter{Исследовательская часть}

В данном разделе произведено сравнение  алгоритмов.

\section{Технические характеристики}

Технические характеристики устройства, на котором выполнялись замеры времени:

\begin{itemize}
	\item операционная система --- Ubuntu 22.04.1 Linux x86\_64;
	\item оперативная память --- 8 ГБ;
	\item процессор --- AMD Ryzen 5 3550H \cite{amd}.
\end{itemize}

Замеры проводились на ноутбуке, включенном в сеть электропитания. Во время замеров ноутбук не был нагружен сторонними приложениями.

\section{Время выполнения алгоритмов}

Замеры проводились для квадратных матриц одной размерности.
На рисунке \ref{img:g1} представлен график, иллюстрирующий зависимость времени работы от размерности матриц для стандартного алгоритма и алгоритма Копперсмита-Винограда.

\begin{figure}[h!]
	\centering
	\begin{tikzpicture}
		\begin{axis}[	
			height = 0.4\paperheight, 
			width = 0.65\paperwidth,
			legend pos = north west,
			table/col sep=comma,
			xlabel={размерность матриц},
			ylabel={время, мкс},
			]
			\legend{ 
				Наивный алг., 
				Вин. без опт., 
				Вин. с опт. +=,
				Вин. с опт. <<,
				Вин. опт.,
			};
			\addplot [
			solid, 
			draw = blue,
			mark = *, 
			mark options = {
				scale = 1.5, 
				fill = blue, 
				draw = black
			}
			] table [x={size}, y={time}] {st.csv};
			\addplot [
			dotted, 
			draw = red,
			mark = star, 
			mark options = {
				scale = 1.5, 
				draw = red
			}
			] table [x={size}, y={time}] {v0.csv};
			\addplot [
			dashed, 
			draw = green,
			mark = x, 
			mark options = {
				scale = 1.5, 
				draw = green
			}
			] table [x={size}, y={time}] {v1.csv};
			\addplot [
			densely dotted, 
			draw = yellow,
			mark = oplus, 
			mark options = {
				scale = 1.5, 
				draw = yellow
			}
			] table [x={size}, y={time}] {v2.csv};
			\addplot [
			loosely dashdotted, 
			draw = violet,
			mark = star, 
			mark options = {
				scale = 1.5, 
				draw = violet
			}
			] table [x={size}, y={time}] {v3.csv};
		\end{axis}
	\end{tikzpicture}
	\caption{Сравнение алгоритмов умножения матриц}
	\label{img:g1}
\end{figure}

\clearpage
\section*{Вывод}

Алгоритм Копперсмита-Винограда превосходит по времени работы стандартный алгоритм на 85\% (для матриц размерности 700 $\cdot$ 700). 

Результаты замеров доказывают важность оптимизаций --- версия алгоритма Копперсмита-Винограда с предварительными вычислениями быстрее обычной на 46\% (для матриц размерности 300 $\cdot$ 300).


