\chapter{Конструкторская часть}
В этом разделе приведены схемы алгоритмов и выполнена оценка трудоёмкости.

\section{Разработка алгоритмов}

Предполагается, что на вход всех алгоритмов поступили матрицы верного размера.

На рисунке \ref{img:St} приведена схема стандартного алгоритма умножения матриц.
\img{170mm}{40mm}{St}{Схема стандартного алгоритма умножения матриц}
\clearpage

На рисунке \ref{img:Vin} приведена схема алгоритма умножения матриц по Винограду.
\img{170mm}{100mm}{Vin}{Схема алгоритма умножения матриц по Винограду}
\clearpage

На рисунке \ref{img:Vin} приведена схема алгоритма умножения матриц по Винограду с оптимизациями.
\img{170mm}{100mm}{Vin}{Схема алгоритма умножения матриц по Винограду с оптимизациями}
\clearpage

\section{Оценка трудоемкости алгоритмов}

\subsection{Модель вычислений}

Для оценки трудоёмкости алгоритмов введём модель вычислений.
\begin{enumerate}
	\item Операции из списка (\ref{model:opers}) имеют трудоемкость 1.
	\begin{equation}
		\label{model:opers}
		+, -, ++, {-}-, *, /, \%, ==, !=, <, >, <=, >=, []
	\end{equation}
	\item Трудоемкость оператора выбора if условие then A else B рассчитывается, как (\ref{model:if}).
	\begin{equation}
		\label{model:if}
		f_{if} = f_{\text{условия}} +
		\begin{cases}
			f_A, & \text{если условие выполняется,}\\
			f_B, & \text{иначе.}
		\end{cases}
	\end{equation}
	\item Трудоемкость цикла рассчитывается, как (\ref{model:cycle}).
	\begin{equation}
		\label{model:cycle}
		\begin{gathered}
		f_{\text{цикла}} = f_{\text{инициализации}} + f_{\text{сравнения}} + \\
		+ N(f_{\text{тела}} + f_{\text{инкремента}} + f_{\text{сравнения}})
		\end{gathered}
	\end{equation}
	\item Трудоемкость вызова функции равна 0.
\end{enumerate}

\subsection{Стандартный алгоритм умножения \newline матриц}

Трудоёмкость стандартного алгоритма умножения матриц складывается из:
\begin{itemize}
	\item трудомкости внешнего цикла по $i \in [1..M]$: $f = 2 + M \cdot (2 + f_{body})$;
	\item трудомкости цикла по $j \in [1..N]$: $f = 2 + N \cdot (2 + f_{body})$;
	\item трудомкости цикла по $k \in [1..Q]$: $f = 2 + 11Q$.
\end{itemize}

Общая трудоёмкость стандартного алгоритма умножения матриц:
\begin{equation}
	f_{st} = 2 + M\cdot(4 + N\cdot(4 + 11Q)) = 2 + 4M + 4MN + 11MNQ,
\end{equation}
\begin{equation}
	f_{st} \approx 11MNQ.
\end{equation}

\subsection{Алгоритм Копперсмита-Винограда}

Вычислим трудоёмкость алгоритма Копперсмита-Винограда:
\begin{enumerate}
	\item трудоёмкость предобработки строк
	\begin{equation}
	\begin{gathered}
		f_{mulH} = \underbrace{1}_{=} + \underbrace{2}_{init} + M \cdot (\underbrace{2}_{inc} + \underbrace{4}_{init} + \frac{N}{2} \cdot \\
		\cdot (\underbrace{4}_{inc} + \underbrace{6}_{[]} + \underbrace{2}_{+} +\underbrace{6}_{*} + \underbrace{1}_{=})),
	\end{gathered}
	\end{equation}
	\begin{equation}
			f_{mulH} = 3 + 6 \cdot M + 9.5 \cdot MN;
	\end{equation}
	
	\item трудоёмкость предобработки столбцов (аналогично строкам)
	\begin{equation}
		f_{mulV} = 3 + 6 \cdot N + 9.5 \cdot MN;
	\end{equation}
	
	\item трудоёмкость внутреннего цикла
	\begin{equation}
	\begin{gathered}
		f_{in} = \underbrace{4}_{init} + \frac{M}{2} \cdot (\underbrace{4}_{inc} + \underbrace{1}_{=} + \underbrace{12}_{[]} + \underbrace{5}_{+} + \underbrace{8}_{*}),
	\end{gathered}
	\end{equation}
	\begin{equation}
		f_{in} = 4 + 15 \cdot M;
	\end{equation}
	
	\item трудоёмкость внешнего цикла
	\begin{equation}
		\begin{gathered}
			f_{out} = \underbrace{2}_{init} + N \cdot (\underbrace{2}_{inc} + \underbrace{2}_{init} + Q \cdot \\ 
			\cdot (\underbrace{2}_{inc} + \underbrace{1}_{=} + \underbrace{4}_{[]} + \underbrace{2}_{+} + f_{in})),
		\end{gathered}
	\end{equation}
	\begin{equation}
		f_{out} = 2 + 4 \cdot N + 13 \cdot NQ + 15 \cdot MNQ;
	\end{equation}

	\item дополнительная трудоёмкость для худшего случая (кол-во\\
		столбцов 1-й матрицы нечётно)
	\begin{equation}
	\begin{gathered}
		f_{add} = \underbrace{2}_{init} + N \cdot (\underbrace{2}_{inc} + \underbrace{2}_{init} + Q \cdot \\
		\cdot (\underbrace{2}_{inc} + \underbrace{1}_{=} + \underbrace{8}_{[]} + \underbrace{3}_{+} + \underbrace{2}_{*})),
	\end{gathered}
	\end{equation}
	\begin{equation}
		f_{add} = 2 + 4N + 16NQ.
	\end{equation}
\end{enumerate}

\clearpage
Таким образом, трудоёмкость всего алгоритма в лучшем случае (кол-во столбцов 1-й матрицы чётно):
\begin{equation}
	f_{best} = f_{mulH} + f_{mulV} + f_{out},
\end{equation}
\begin{equation}
	f_{best} = 8 + 6M + 10N + 19MN + 13 NQ + 15 MNQ.
\end{equation}

Трудоёмкость всего алгоритма в худшем случае:
\begin{equation}
	f_{worst} = f_{best} + f_{add},
\end{equation}
\begin{equation}
	f_{worst} = 10 + 6M + 14N + 19MN + 29NQ + 15 MNQ.
\end{equation}

\subsection{Оптимизированный алгоритм \newline Копперсмита-Винограда}

Алгоритм Копперсмита-Винограда можно оптимизировать следующим образом:
\begin{itemize}
	\item заменить операцию x = x + k на x += k;
	\item заменить умножение на 2 на побитовый сдвиг;
	\item предвычислять некоторые слагаемые для алгоритма.
\end{itemize}

Вычислим трудоёмкость оптимизированного алгоритма Копперсмита-Винограда:
\begin{enumerate}
	\item трудоёмкость предобработки строк
	\begin{equation}
		\begin{gathered}
			f_{mulH} = \underbrace{1}_{=} + \underbrace{2}_{init} + M \cdot (\underbrace{2}_{inc} + \underbrace{2}_{init} + \frac{N}{2} \cdot \\
			\cdot (\underbrace{2}_{inc} + \underbrace{1}_{-=} \underbrace{5}_{[]} + \underbrace{1}_{-} +\underbrace{2}_{*} + )),
		\end{gathered}
	\end{equation}
	\begin{equation}
		f_{mulH} = 3 + 4 \cdot M + 5.5 \cdot MN;
	\end{equation}
	
	\item трудоёмкость предобработки столбцов (аналогично строкам)
	\begin{equation}
		f_{mulV} = 3 + 4 \cdot N + 5.5 \cdot MN;
	\end{equation}
	
	\item трудоёмкость внутреннего цикла
	\begin{equation}
		\begin{gathered}
			f_{in} = \underbrace{2}_{init} + \frac{N}{2} \cdot (\underbrace{2}_{inc} + \underbrace{1}_{+=} + \underbrace{8}_{[]} + \underbrace{4}_{+/-} + \underbrace{2}_{*}),
		\end{gathered}
	\end{equation}
	\begin{equation}
		f_{in} = 2 + 8.5 \cdot N;
	\end{equation}
	
	\item трудоёмкость внешнего цикла
	\begin{equation}
		\begin{gathered}
			f_{out} = \underbrace{2}_{init} + M \cdot (\underbrace{2}_{inc} + \underbrace{2}_{init} + Q \cdot \\ 
			\cdot (\underbrace{2}_{inc} + \underbrace{2}_{=} + \underbrace{4}_{[]} + \underbrace{1}_{+} + f_{in})),
		\end{gathered}
	\end{equation}
	\begin{equation}
		f_{out} = 2 + 4 \cdot M + 11 \cdot MQ + 8.5 \cdot MNQ;
	\end{equation}
	
	\item дополнительная трудоёмкость для худшего случая (кол-во\\
	столбцов 1-й матрицы нечётно)
	\begin{equation}
		\begin{gathered}
			f_{add} = \underbrace{2}_{init} + N \cdot (\underbrace{2}_{inc} + \underbrace{2}_{init} + Q \cdot \\
			\cdot (\underbrace{2}_{inc} + \underbrace{1}_{=} + \underbrace{8}_{[]} + \underbrace{3}_{+} + \underbrace{2}_{*})),
		\end{gathered}
	\end{equation}
	\begin{equation}
		f_{add} = 2 + 4N + 16NQ.
	\end{equation}
\end{enumerate}

Таким образом, трудоёмкость всего алгоритма в лучшем случае (кол-во столбцов 1-й матрицы чётно):
\begin{equation}
	f_{best} = f_{mulH} + f_{mulV} + f_{out},
\end{equation}
\begin{equation}
	f_{best} = 8 + 6M + 10N + 19MN + 13 NQ + 15 MNQ.
\end{equation}

Трудоёмкость всего алгоритма в худшем случае:
\begin{equation}
	f_{worst} = f_{best} + f_{add},
\end{equation}
\begin{equation}
	f_{worst} = 10 + 6M + 14N + 19MN + 29NQ + 15 MNQ.
\end{equation}









