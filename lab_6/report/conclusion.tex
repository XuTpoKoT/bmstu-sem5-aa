\chapter*{Заключение}
\addcontentsline{toc}{chapter}{Заключение}

Цель достигнута: были описаны и реализованы решения задачи коммивояжера с помощью полного перебора и с помощью муравьиного алгоритма.
В ходе выполнения лабораторной работы были решены все задачи:
\begin{enumerate}
	\item описаны и реализованы алгоритм полного перебора и муравьиный алгоритм;
	\item проведена параметризация муравьиного алгоритма на двух классах данных;
	\item проведено сравнение затрат реализаций по времени выполнения.
\end{enumerate}

Были также сделаны выводы на основе полученных данных.
Эвристический метод, основанный на муравьином алгоритме имеет преимущество перед методом полного перебора за счет того, что способен работать с данными достаточно большого объема, в то время как полный перебор сильно ограничен размером данных.
Также были подобраны параметры для оптимальной работы метода на двух классах данных.
Однако, в отличии от полного перебора, эвристический алгоритм не гарантирует точность найденного им пути, есть вероятность, что путь будет не оптимален.



