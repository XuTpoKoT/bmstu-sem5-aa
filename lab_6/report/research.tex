\chapter{Исследовательская часть}

\section{Параметризация метода}
В муравьином алгоритме вычисления производятся на основе настраиваемых параметров.
Рассмотрим два класса данных и подберем к ним параметры, при которых метод даст точный результат при минимальном количестве итераций (итерациями считаются кол-во дней).
Будем рассматривать матрицы размерности $10\times10$ и проводить 10 запусков для каждого набора параметров.

В качестве первого класса данных рассмотрим графы, матрицы смежности которых содержат значения в диапазоне $[1, 5]$.

В качестве второго класса данных рассмотрим графы, матрицы смежности которых содержат значения в диапазоне $[1, 1000]$.

Результат параметризации представлен в таблицах \ref{tbl:1}--\ref{tbl:6} со значениями $a$, $p$, $tmax$, $delta$, где
\begin{itemize}
	\item $a$, $p$ --- настраиваемые параметры;
	\item $tmax$ --- число дней симуляции;
	\item $delta$ --- разность эталонного решения задачи и решения муравьиного алгоритма с данными параметрами на первом классе данных.
\end{itemize}

Наилучшие параметры для класса данных 1: $a = 0.9, p = 0.8, tmax = 200.$

Наилучшие параметры для класса данных 2: $a = 0.1, p = 0.2, tmax = 50.$

\begin{table}[h!]
	\caption{Результат параметризации для класса данных 1 (ч. 1)}
	\label{tbl:1}
	\begin{center}
		\csvautotabular{graph1_1.csv}
	\end{center}
\end{table}
\newpage 

\begin{table}[h!]
	\caption{Результат параметризации для класса данных 1 (ч. 2)}
	\label{tbl:2}
	\begin{center}
		\csvautotabular{graph1_2.csv}
	\end{center}
\end{table}
\newpage 

\begin{table}[h!]
	\caption{Результат параметризации для класса данных 1 (ч. 3)}
	\label{tbl:3}
	\begin{center}
		\csvautotabular{graph1_3.csv}
	\end{center}
\end{table}
\newpage 

\begin{table}[h!]
	\caption{Результат параметризации для класса данных 2 (ч. 1)}
	\label{tbl:4}
	\begin{center}
		\csvautotabular{graph2_1.csv}
	\end{center}
\end{table}
\newpage 

\begin{table}[h!]
	\caption{Результат параметризации для класса данных 2 (ч. 2)}
	\label{tbl:5}
	\begin{center}
		\csvautotabular{graph2_2.csv}
	\end{center}
\end{table}
\newpage 

\begin{table}[h!]
	\caption{Результат параметризации для класса данных 2 (ч. 3)}
	\label{tbl:6}
	\begin{center}
		\csvautotabular{graph2_3.csv}
	\end{center}
\end{table}
\clearpage 

\section{Технические характеристики}

Технические характеристики устройства, на котором выполнялись замеры времени:

\begin{itemize}
	\item операционная система --- Ubuntu 22.04.1 Linux x86\_64;
	\item оперативная память --- 8 ГБ;
	\item процессор --- AMD Ryzen 5 3550H \cite{amd}.
\end{itemize}

Замеры проводились на ноутбуке, включенном в сеть электропитания.
Во время замеров ноутбук не был нагружен сторонними приложениями.

\section{Время выполнения алгоритмов}

Для проведения замеров времени, программа запускалась на различных размерах графов.
Реализация муравьиного алгоритма работала при лучших параметриах.

На рисунке \ref{img:g1} представлены результаты замеров времени выполнения реализаций алгоритмов для класса данных 1.

\begin{figure}[h!]
	\centering
	\begin{tikzpicture}
		\begin{axis}[	
			height = 0.4\paperheight,
			width = 0.65\paperwidth,
			legend pos = north west,
			table/col sep=comma,
			xlabel={кол-во городов},
			ylabel={время, нс},
			]
			\legend{ 
				Полный перебор, 
				Мур. алгоритм, 
			};
			\addplot [
			solid, 
			draw = blue,
			mark = *, 
			mark options = {
				scale = 1.5, 
				fill = blue, 
				draw = black
			}
			] table [x={size}, y={time}] {t_ex1.csv};
			\addplot [
			dotted, 
			draw = red,
			mark = star, 
			mark options = {
				scale = 1.5, 
				draw = red
			}
			] table [x={size}, y={time}] {t_ant1.csv};		
		\end{axis}
	\end{tikzpicture}
	\caption{Время выполнения реализаций алгоритмов для класса данных 1}
	\label{img:g1}
\end{figure}
\clearpage

На рисунке \ref{img:g2} представлены результаты замеров времени выполнения реализаций алгоритмов для класса данных 2.

\begin{figure}[h!]
	\centering
	\begin{tikzpicture}
		\begin{axis}[	
			height = 0.4\paperheight, 
			width = 0.65\paperwidth,
			legend pos = north west,
			table/col sep=comma,
			xlabel={кол-во городов},
			ylabel={время, нс},
			]
			\legend{ 
				Полный перебор, 
				Мур. алгоритм, 
			};
			\addplot [
			solid, 
			draw = blue,
			mark = *, 
			mark options = {
				scale = 1.5, 
				fill = blue, 
				draw = black
			}
			] table [x={size}, y={time}] {t_ex2.csv};
			\addplot [
			dotted, 
			draw = red,
			mark = star, 
			mark options = {
				scale = 1.5, 
				draw = red
			}
			] table [x={size}, y={time}] {t_ant2.csv};
		\end{axis}
	\end{tikzpicture}
	\caption{Время выполнения реализаций алгоритмов для класса данных 2}
	\label{img:g2}
\end{figure}

\clearpage
\section*{Вывод}
Для обоих классов данных при кол-ве городов до 9 время работы реализации муравьиного алгоритма больше, чем при полном переборе: для 8 городов в 3 раза больше на графе первого класса данных и в 1.5 раз больше на графе второго класса.
При большем кол-ве городов  реализация муравьиного алгоритма выигрывает по временной эффективности: в 8 раз для графа с 10 вершинами класса данных 1 и в 30 раз для графа той же размерности класса данных 2.

