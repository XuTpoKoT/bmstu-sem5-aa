\chapter{Технологическая часть}

\section{Средства реализации}

Для реализации данной лабораторной работы был выбран язык программирования С++ \cite{c} и среда разработки CLion, которая позволяет замерять процессорное время с помощью пакета \texttt{<ctime>} \cite{ctime}.

\section{Реализация алгоритмов}

В листингах \ref{lst:ex}--\ref{lst:ant} приведены реализации класса Colony, алгоритма полного перебора и муравьиного алгоритма.
В листингах \ref{lst:makePath}--\ref{lst:fer} приведены реализации вспомогательных функций.

\begin{lstlisting}[label=lst:ex,caption=Алгоритм полного перебора]
std::pair<std::vector<int>, double> exhaustiveSearch(const Graph &graph) {
	std::vector<int> bestPath(graph.size());
	std::vector<int> curPath(graph.size());
	for (int i = 0; i < graph.size(); i++) {
		curPath[i] = i;
	}
	double bestPathLen = pathLen(graph, curPath);
	
	while(std::next_permutation(curPath.begin() + 1, curPath.end())) {
		auto len = pathLen(graph, curPath);
		if (len < bestPathLen) {
			bestPathLen = len;
			bestPath = curPath;
		}
	}
	
	return std::make_pair(bestPath, bestPathLen);
}
\end{lstlisting}

\clearpage
\begin{lstlisting}[label=lst:col,caption=Класс Colony]
struct SimulationResult {
	std::vector<size_t> path;
	double pathLen;
};

class Colony {
	private:
	static constexpr const double START_PHEROMONE = 0.3;
	std::default_random_engine generator;
	std::uniform_real_distribution<double> distribution = std::uniform_real_distribution<double>(0, 1);
	
	Graph graph;
	Graph pheromoneGraph;
	Parameters params;
	
	public:
	explicit Colony(Graph &graph);
	SimulationResult simulation(Parameters parameters);
	
	private:
	std::vector<size_t> makePath(size_t start_vert);
	double getPathLen(std::vector<size_t> path);
	void updatePheromone(const std::vector<std::vector<size_t>> &paths);
	void vaporizePheromone();
	void addPheromone(const std::vector<std::vector<size_t>> &paths);
	void addPheromone(const std::vector<size_t> &path);
	double randomProbability();
	std::vector<double> vertexesProbabilities(size_t curVert, const std::vector<size_t> &availableVertexes);
	size_t choseVert(size_t vertex, const std::vector<size_t> &visitedVertexes);
};
\end{lstlisting}

\clearpage
\begin{lstlisting}[label=lst:ant,caption=Муравьиный алгоритм]
SimulationResult Colony::simulation(Parameters parameters) {
	params = parameters;
	pheromoneGraph = Graph(graph.size(), START_PHEROMONE);
	std::vector<size_t> bestPath;
	double bestPathLen = -1;
	
	for (size_t i = 0; i < params.simulationDays; ++i) {
		std::vector<std::vector<size_t>> dayPaths(graph.size());
		for (size_t v = 0; v < graph.size(); ++v) {
			auto path = makePath(v);
			auto pathLen = getPathLen(path);
			dayPaths.at(v) = path;
			
			if (pathLen < bestPathLen || std::abs(bestPathLen + 1) < std::numeric_limits<double>::epsilon()) {
				bestPath = path;
				bestPathLen = pathLen;
			}
		}
		updatePheromone(dayPaths);
		addPheromone(bestPath); // elite ants
	}
	
	return { bestPath, bestPathLen };
}
\end{lstlisting}

\begin{lstlisting}[label=lst:makePath,caption=Построение маршрута]
std::vector<size_t> Colony::makePath(size_t startVert) {
	std::vector<size_t> visitedVertexes = {startVert};
	
	auto curVert = startVert;
	while (visitedVertexes.size() != graph.size()) {
		curVert = choseVert(curVert, visitedVertexes);
		visitedVertexes.push_back(curVert);
	}
	
	return visitedVertexes;
}
\end{lstlisting}

\begin{lstlisting}[label=lst:nextV,caption=выбор следующей вершины]
size_t Colony::choseVert(size_t vertex, const std::vector<size_t> &visitedVertexes) {
	auto probability = randomProbability();
	auto availableVertexes = graph.getAvailableVertices(vertex, visitedVertexes);
	
	auto probabilities = vertexesProbabilities(vertex, availableVertexes);
	auto denominator = std::accumulate(probabilities.begin(), probabilities.end(), 0.0);
	
	double curProb = 0.0;
	for (auto &v: availableVertexes) {
		curProb += probabilities.at(v) / denominator;
		if (curProb >= probability) {
			return v;
		}
	}
	
	return availableVertexes[availableVertexes.size() - 1];
}
\end{lstlisting}

\begin{lstlisting}[label=lst:prob,caption=получение вероятностей перехода в доступные вершины]
std::vector<double> Colony::vertexesProbabilities(size_t curVert, const std::vector<size_t> &availableVertexes) {
	std::vector<double> probabilities(graph.size(), 0);
	
	for (auto &v: availableVertexes) {
		probabilities.at(v) = static_cast<double &&>(
		std::pow(pheromoneGraph.get(curVert, v), params.a)
		* std::pow(1. / graph.get(curVert, v), params.b)
		);
	}
	
	return probabilities;
}
\end{lstlisting}

\clearpage
\begin{lstlisting}[label=lst:fer,caption=обновление феромона]
void Colony::updatePheromone(const std::vector<std::vector<size_t>> &paths) {
	for (size_t i = 0; i < graph.size(); ++i) {
		for (size_t j = 0; j < graph.size(); ++j) {
			if (i == j) {
				continue;
			}
			
			pheromoneGraph.set(i, j, pheromoneGraph.get(i, j) * (1 - params.p));
			if (pheromoneGraph.get(i, j) <= 1e-5) {
				pheromoneGraph.set(i, j, 0.3);
			}
		}
	}

	for (auto &path: paths) {
		for (size_t i = 0; i < path.size() - 1; ++i) {
			size_t from = path.at(i);
			size_t to = path.at(i + 1);
			auto delta = params.q / graph.get(from, to);
			auto new_val = pheromoneGraph.get(from, to) + delta;
			pheromoneGraph.set(from, to, new_val);
			pheromoneGraph.set(to, from, new_val);
		}
	}
}
\end{lstlisting}
\clearpage


\section{Функциональное тестирование}
В таблице~\ref{tabular:test_rec} приведены тесты для функций, реализующих алгоритм полного перебора и муравьиный алгоритм.
Все тесты пройдены успешно.

\begin{table}[h!]
	\begin{center}
		
		\caption{\label{tabular:test_rec} Тестирование функций}
		\begin{tabular}{c@{\hspace{5mm}}c@{\hspace{5mm}}c@{\hspace{5mm}}c@{\hspace{7mm}}c@{\hspace{7mm}}c@{\hspace{7mm}}}
			\hline
			Матрица смежности & Ожидаемый рез. & Действительный рез. \\ \hline
			\vspace{4mm}
			$\begin{pmatrix}
				0 &  1 &  10 &  7\\
				1 &  0 &  1 &  2\\
				10 &  1 &  0 &  1\\
				7 &  2 &  1 &  0
			\end{pmatrix}$ &
			10, [0, 1, 2, 3, 0]&
			10, [0, 1, 2, 3, 0]\\
			\vspace{2mm}
			\vspace{2mm}
			$\begin{pmatrix}
				0 &  3 &  5 &  7\\
				7 &  0 &  1 &  2\\
				5 &  1 &  0 &  1\\
				7 &  2 &  1 &  0
			\end{pmatrix}$ &
			11, [2, 3, 1, 0, 2]&
			11, [2, 3, 1, 0, 2]\\
			\vspace{2mm}
			\vspace{2mm}
			$\begin{pmatrix}
				0 &  3 &  5\\
				3 &  0 &  1\\
				5 &  1 &  0
			\end{pmatrix}$ &
			9, [0, 1, 2, 0]&
			9, [0, 1, 2, 0]\\
		\end{tabular}
	\end{center}
\end{table}