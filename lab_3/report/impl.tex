\chapter{Технологическая часть}

В данном разделе представлены требования к программному обеспечению, средства реализации, листинги кода и проведена оценка трудоёмкости.

\section{Требования к программному \newline обеспечению}

Используемое программное обеспечение должно предоставлять возможность измерения процессорного времени.

\section{Средства реализации}

Для реализации данной лабораторной работы был выбран язык программирования С++ \cite{c} и среда разработки CLion, которая позволяет замерять процессорное время с помощью пакета \texttt{<ctime>} \cite{ctime}.

\section{Реализация алгоритмов}

В листингах \ref{lst:shakerSort}--\ref{lst:treeSort} приведены реализации алгоритмов, в листингах \ref{lst:insert}--\ref{lst:trav} --- функции вставки в дерево и его обхода.

\clearpage
\begin{lstlisting}[label=lst:shakerSort,caption=Функция сортировки перемешиванием]
void shakerSort(std::vector<int> &a) {
	int left = 0;
	int right = a.size() - 1;
	int newBorder = left;
	
	while (right > left) {
		for (int i = left; i < right; i++) {
			if (a[i] > a[i+1]) {
				int tmp = a[i];
				a[i] = a[i+1];
				a[i+1] = tmp;
				newBorder = i;
			}
		}
		right = newBorder;
		
		for (int i = right; i > left; i--) {
			if (a[i-1] > a[i]) {
				int tmp = a[i];
				a[i] = a[i-1];
				a[i-1] = tmp;
				newBorder = i;
			}
		}
		left = newBorder;
	}
}	
\end{lstlisting}

\clearpage
\begin{lstlisting}[label=lst:gnomeSort,caption=Функция гномьей сортировки]
void gnomeSort(std::vector<int> &a) {
	int n =  a.size();
	
	for (int i = 0; i < n;) {
		if (i == 0 || a[i-1] <= a[i]) {
			i++;
		} else {
			int tmp = a[i];
			a[i] = a[i-1];
			a[i-1] = tmp;
			i--;
		}
	}
}	
\end{lstlisting}

\begin{lstlisting}[label=lst:treeSort,caption=Функция сортировки двоичным деревом]
void treeSort(std::vector<int> &arr) {
	auto tree = std::make_unique<BSTree>();
	
	for (int &el : arr) {
		tree->insert(el);
	}
	
	tree->traverse(arr);
}	
\end{lstlisting}
\clearpage
\begin{lstlisting}[label=lst:insert,caption=Функция вставки в дерево]
BinTreeNode* insert(BinTreeNode *node, int data) {
	if (node == nullptr) {
		node = new BinTreeNode{data, nullptr, nullptr};
	} else if (data < node->data) {
		node->left = insert(node->left, data);
	} else {
		node->right = insert(node->right, data);
	}
	return node;
}	
\end{lstlisting}

\begin{lstlisting}[label=lst:trav,caption=Функция обхода дерева]
void traverse(BinTreeNode *node, std::vector<int> &arr, int &index) {
	if (node->left != nullptr) {
		traverse(node->left,arr, index);
	}
	
	arr[index] = node->data;
	index++;
	
	if (node->right != nullptr) {
		traverse(node->right,arr, index);
	}
}	
\end{lstlisting}
\clearpage

\section{Оценка трудоемкости алгоритмов}

\subsection{Модель вычислений}

Для оценки трудоёмкости алгоритмов введём модель вычислений.
\begin{enumerate}
	\item Операции из списка (\ref{model:opers}) имеют трудоемкость 1.
	\begin{equation}
		\label{model:opers}
		+, -, ++, {-}-, *, /, \%, ==, !=, <, >, <=, >=, []
	\end{equation}
	\item Трудоемкость оператора выбора if условие then A else B рассчитывается, как (\ref{model:if}).
	\begin{equation}
		\label{model:if}
		f_{if} = f_{\text{условия}} +
		\begin{cases}
			f_A, & \text{если условие выполняется;}\\
			f_B, & \text{иначе.}
		\end{cases}
	\end{equation}
	\item Трудоемкость цикла с $N$ итерациями рассчитывается, как (\ref{model:cycle}).
	\begin{equation}
		\label{model:cycle}
		\begin{gathered}
			f_{\text{цикла}} = f_{\text{инициализации}} + f_{\text{сравнения}} + \\
			+ N(f_{\text{тела}} + f_{\text{инкремента}} + f_{\text{сравнения}})
		\end{gathered}
	\end{equation}
	\item Трудоемкость вызова функции равна 0.
\end{enumerate}

\subsection{Cортировка перемешиванием}

Рассчитаем трудоёмкость для худшего случая --- массив отсортирован в обратном порядке.
\begin{enumerate}
	\item Трудоёмкость условного оператора:
	\begin{equation}
		\begin{gathered}
			f_{if} = \underbrace{4}_{cmp} + \underbrace{2+4+3}_{swap} + \underbrace{1}_{newborder} + \underbrace{2}_{inc} + \underbrace{2}_{cmp} = 18.
		\end{gathered}
	\end{equation}
	\item Трудоёмкость внешнего цикла:
	\begin{equation}
		\begin{gathered}z
			f_{out} = \underbrace{4}_{init} + \frac{N}{2} \cdot (\underbrace{1}_{cmp} + 	\underbrace{2}_{inc} + (N-1) \cdot (\underbrace{2}_{init} + \underbrace{2}_{cmp})).
		\end{gathered}
	\end{equation}
	\item Общая трудоёмкость в худшем случае:
	\begin{equation}
		\begin{gathered}
			f_{worst} = 4 - 8 \cdot N + 11 \cdot N^2.
		\end{gathered}
	\end{equation}
\end{enumerate}	

Трудоёмкость алгоритма сортировки пермешиванием в худшем случае оценена как $O(N^2)$.

Рассчитаем трудоёмкость для лучшего случая --- массив уже отсортирован.
\begin{enumerate}
	\item Трудоёмкость условного оператора:
	\begin{equation}
		\begin{gathered}
			f_{if} = \underbrace{4}_{cmp}.
		\end{gathered}
	\end{equation}
	\item Трудоёмкость первого внутреннего цикла:
	\begin{equation}
		\begin{gathered}
			f_{in1} = \underbrace{2}_{init} + N \cdot (\underbrace{1}_{cmp} + 	\underbrace{1}_{inc} + f_{if}).
		\end{gathered}
	\end{equation}
	\item Трудоёмкость второго внутреннего цикла:
	\begin{equation}
		\begin{gathered}
			f_{in2} = \underbrace{2}_{init}.
		\end{gathered}
	\end{equation}
	\item Трудоёмкость внешнего цикла:
	\begin{equation}
		\begin{gathered}
			f_{out} = \underbrace{4}_{init} + \underbrace{1}_{cmp} + \underbrace{2}_{inc} + f_{in1} + f_{in2}.
		\end{gathered}
	\end{equation}
	\item Общая трудоёмкость в лучшем случае:
	\begin{equation}
		\begin{gathered}
			f_{best} =6 \cdot N + 11 \approx 6 \cdot N.
		\end{gathered}
	\end{equation}
\end{enumerate}	

\clearpage
\subsection{Гномья сортировка}
Рассчитаем трудоёмкость для худшего случая --- массив отсортирован в обратном порядке.
\begin{enumerate}
	\item Трудоёмкость проверки условия:
	\begin{equation}
		\begin{gathered}
			f_{cmp} = 6.
		\end{gathered}
	\end{equation}
	\item Трудоёмкость условного оператора при истинном условии:
	\begin{equation}
		\begin{gathered}
			f_{if} = f_{cmp} + 1 = 7.
		\end{gathered}
	\end{equation}
	\item Трудоёмкость условного оператора при ложном условии:
	\begin{equation}
		\begin{gathered}
			f_{else} = f_{cmp} + 10 = 16.
		\end{gathered}
	\end{equation}
	
	\item Трудоёмкость цикла:
	\begin{equation}
		\begin{gathered}
			f_{out} = \underbrace{2}_{init} + \frac{N - 1}{2} \cdot N \cdot (\underbrace{1}_{cmp} + f_{if} + f_{else}) + N \cdot f_{if}.
		\end{gathered}
	\end{equation}
	\item Общая трудоёмкость в худшем случае:
	\begin{equation}
		\begin{gathered}
			f_{worst} = 12 \cdot N^2 - 5 \cdot N + 2 \approx 12 \cdot N^2.
		\end{gathered}
	\end{equation}
\end{enumerate}	

\clearpage
Рассчитаем трудоёмкость для лучшего случая --- массив уже отсортирован.
\begin{enumerate}
	\item Трудоёмкость условного оператора:
	\begin{equation}
		\begin{gathered}
			f_{if} = 7.
		\end{gathered}
	\end{equation}
	\item Общая трудоёмкость в лучшем случае:
	\begin{equation}
		\begin{gathered}
			f_{best} = N \cdot f_{if} = 7 \cdot N.
		\end{gathered}
	\end{equation}
\end{enumerate}

\subsection{Cортировка двоичным деревом}
Трудоёмкость этого алгоритма в лучшем случае, когда вставка каждого элемента производится в сбалансированное дерево, пропорциональна $N \cdot \log{N}$, в худшем случае --- для невозрастающей или неубывающей последовательности --- $N^2$ \cite{knuth}.


