\chapter{Аналитическая часть}
В этом разделе представлены описания алгоритмов.

\section{Гномья сортировка}
В начеле алгоритма создаётся указатель на второй элемент массива. После этого происходит cравнение текущего и предыдущего элементов. Если порядок соблюден, происходит переход к следующему элементу, если нет, то элементы меняются местами и указатель в цикле переходит к предыдущему элементу. Цикл сортировки заканчивается в тот момент, когда номер указателя становится равным длине массива \cite{shakerGnome}. 

\section{Сортировка перемешиванием}
По данному алгоритму в каждой итерации цикла указатель перемещается от левого края массива до правого и при нарушении порядка элементов они меняются местами. После прохода слева направо указатель возвращается к началу массива, так же при нарушении порядка элементов меняя их \cite{shakerGnome}. 

\section{Сортировка двоичным деревом}
Сортировка с помощью двоичного дерева --- универсальный алгоритм сортировки, заключающийся в построении двоичного дерева поиска по ключам массива (списка), с последующей сборкой результирующего массива путём обхода узлов построенного дерева в необходимом порядке следования ключей. То есть, если при обходе дерева записывать все встречающиеся элементы в массив, получится упорядоченное в порядке возрастания множество \cite{knuth}.
