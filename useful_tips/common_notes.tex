\chapter{Общие замечания}

\section{Оформление}
\begin{enumerate}
	\item Введение и заключение не номеруются, но есть в оглавлении.
	\item 'Оглавление', а не 'Содержание'; 'Список использованных источников', а не 'Литература';
	\item Подписи к рисункам - снизу, к таблицам - сверху. Формат подписей: Рисунок 1.1 --  <назв.>, Таблица 1.1 -- <назв.>. Рисунки и таблицы должны располагаться сразу после первой формулы, которая на них ссылается.
	\item Подписать оси на графике, отметить точки, линии рисовать не только разного цвета, но и разного типа (сплошная, пунктир и т.п.).
	\item Если листинг не умещается на странице, разбить его на несколько, подписать начало/продолжение/окончание.
	\item В нумерованных списках после цифры ставить скобку, а не точку. Если элементы списка является частью большого предложения, они должны начинаться с маленькой буквы и заканчиваться точкой с запятой (после последнего -- точка).
	\item Пояснения к формулам: в конце формулы запятая, после неё "где" с маленькой буквы и на той же строчке погнал. Неопределённых переменных быть не должно. Формулы не должны вылезать за границы листа (где кончается текст -- там должна кончиться формула). Фразу "следующая формула" заменить на "формула (<номер\_формулы>)", используя \text{ref}, конечно. Номер формулы обернуть в скобки. Формулы могут быть частью предложений.\newline
	\textcolor{red}{Плохо}:\newline
	Вычислим трудоёмкость алгоритма Копперсмита — Винограда:
	\begin{enumerate}
		\item трудоёмкость предобработки строк
		\begin{equation}
			\begin{gathered}
				f_{mulH} = \underbrace{1}_{=} + \underbrace{2}_{init} + M \cdot (\underbrace{2}_{inc} + \underbrace{4}_{init} + \frac{N}{2} \cdot \\
				\cdot (\underbrace{4}_{inc} + \underbrace{6}_{[]} + \underbrace{2}_{+} +\underbrace{6}_{*} + \underbrace{1}_{=}))
			\end{gathered}
		\end{equation}
		\begin{equation}
			f_{mulH} = 3 + 6 \cdot M + 9.5 \cdot MN
		\end{equation}
	\end{enumerate}
	\textcolor{green}{Лучше}:\newline
	Вычислим трудоёмкость алгоритма Копперсмита — Винограда:
	\begin{enumerate}
		\item трудоёмкость предобработки строк
		\begin{equation}
			\begin{gathered}
				f_{mulH} = \underbrace{1}_{=} + \underbrace{2}_{init} + M \cdot (\underbrace{2}_{inc} + \underbrace{4}_{init} + \frac{N}{2} \cdot \\
				\cdot (\underbrace{4}_{inc} + \underbrace{6}_{[]} + \underbrace{2}_{+} +\underbrace{6}_{*} + \underbrace{1}_{=})),
			\end{gathered}
		\end{equation}
		\begin{equation}
			f_{mulH} = 3 + 6 \cdot M + 9.5 \cdot MN;
		\end{equation}
	\end{enumerate}

\clearpage
	\item Вместо * использовать cdot. Знак умножения между буквами лучше не опускать.\newline
	\textcolor{red}{Плохо}:
	\begin{equation}
		\label{t}
		t = \sum_{i=1}^{n/2} a_{i}a_{i+1} + \sum_{i=1}^{n/2} b_{i}b_{i+1}
	\end{equation}
	\newline
	\textcolor{green}{Лучше}:
	\begin{equation}
		\label{t}
		t = \sum_{i=1}^{n/2} a_{i} \cdot a_{i+1} + \sum_{i=1}^{n/2} b_{i} \cdot b_{i+1}
	\end{equation}

	\item \textcolor{red}{Плохо}:
	\begin{table}[h!]
		\begin{center}
			\caption{\label{ft} Функциональные тесты}
			\begin{tabular}{|c|c|c|c|c|}
				\hline
				& & & \multicolumn{2}{c|}{Ожидаемый результат} \\
				\hline
				№&Строка 1&Строка 2&Левенштейн&Дамерау-Л. \\
				\hline
				1&cat&cute&2&2 \\
				\hline
			\end{tabular}
		\end{center}
	\end{table}
	\newline
	\textcolor{green}{Лучше}:
	\begin{table}[h!]
		\begin{center}
			\caption{\label{ft} Функциональные тесты}
			\begin{tabular}{|c|c|c|c|c|}
				\hline
				& \multicolumn{2}{c|}{Входные данные} & \multicolumn{2}{c|}{Ожидаемый результат} \\
				\hline
				№&Строка 1&Строка 2&Левенштейн&Дамерау-Л. \\
				\hline
				1&cat&cute&2&2 \\
				\hline
			\end{tabular}
		\end{center}
	\end{table}
\end{enumerate}

\clearpage
\section{Содержание отчёта, формулировки}
\begin{enumerate}
	\item Кеш-кэш.
	\item Заключение: обязательно сначала написать фразу "Цель достигнута". Далее по пунктам описать, что именно было сделано.
	\item Субъективщина: "Понятный синтаксис", "удобный интерфейс" и прочая субъективщина -- бан;
	\item Введение и заключение -- это краткий пересказ всей работы (чтобы не читать основную инфу).
	\item В технологической части выдвинуть требования к программе; выбор ЯП обосновать этими требованиями.
	\item Все рекламные эпитеты убрать, т. е. не брать эпитеты с сайта разработчика, а доказать, что "язык может измерять процессорное время".
	\item В заключение включить конкретные числа времени работы алгоритмов.
\end{enumerate}

\section{Схема алгоритма}
\begin{enumerate}
	\item Блоки в схеме алгоритмов бесцветные, одного размера (в yed есть автоматическре выравнивание блок схем).
\end{enumerate}










