\chapter{Исследовательская часть}

\section{Технические характеристики}

Технические характеристики устройства, на котором выполнялись замеры времени:

\begin{itemize}
	\item операционная система Ubuntu 22.04.1 Linux x86\_64;
	\item оперативная память 8 Гбайт;
	\item процессор AMD Ryzen 5 3550H, 8 физических ядер, 8 логических ядер~\cite{amd}.
\end{itemize}

Замеры проводились на ноутбуке, включенном в сеть электропитания. Во время замеров ноутбук не был нагружен сторонними приложениями.

\section{Время выполнения алгоритмов}
На рисунке \ref{img:g1} представлен график, иллюстрирующий зависимость времени работы последовательной и параллельной версии алгоритма от количества потоков.

\clearpage
\begin{figure}[h!]
	\centering
	\begin{tikzpicture}
		\begin{axis}[	
			height = 0.4\paperheight, 
			width = 0.65\paperwidth,
			legend pos = north west,
			table/col sep=comma,
			xlabel={кол-во потоков},
			ylabel={время, мкс},
			]
			\legend{ 
				Последовательная версия, 
				Параллельная версия, 
			};
			\addplot [
			solid, 
			draw = blue,
			mark = *, 
			mark options = {
				scale = 1.5, 
				fill = blue, 
				draw = black
			}
			] table [x={size}, y={time}] {ts.csv};
			\addplot [
			dotted, 
			draw = red,
			mark = star, 
			mark options = {
				scale = 1.5, 
				draw = red
			}
			] table [x={size}, y={time}] {tp.csv};		
		\end{axis}
	\end{tikzpicture}
	\caption{Сравнение реализаций алгоритма по времени работы}
	\label{img:g1}
\end{figure}

\section*{Вывод}
Как видно из полученных данных, параллельная реализация работает быстрее последовательной при количестве рабочих потоков от 2 до 16.
Начиная с 8 рабочих потоков, время работы параллельной реализации растёт, при 32
рабочих потоках оно превышает время работы последовательной.
Это связано с тем, что процессор не может эффективно обслуживать более 8 потоков.
