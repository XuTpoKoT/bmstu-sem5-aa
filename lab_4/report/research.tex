\chapter{Исследовательская часть}

В данном разделе произведено сравнение  алгоритмов.

\section{Технические характеристики}

Технические характеристики устройства, на котором выполнялись замеры времени:

\begin{itemize}
	\item операционная система --- Ubuntu 22.04.1 Linux x86\_64;
	\item оперативная память --- 8 ГБ;
	\item процессор --- AMD Ryzen 5 3550H \cite{amd}.
\end{itemize}

Замеры проводились на ноутбуке, включенном в сеть электропитания. Во время замеров ноутбук не был нагружен сторонними приложениями.

\section{Время выполнения алгоритмов}

На рисунке \ref{img:g1} представлен график, иллюстрирующий зависимость времени работы алгоритмов от размерности массивов при заполнении их случайными данными.

\begin{figure}[h!]
	\centering
	\begin{tikzpicture}
		\begin{loglogaxis}[	
			height = 0.4\paperheight, 
			width = 0.65\paperwidth,
			legend pos = north west,
			table/col sep=comma,
			xlabel={размерность массивов},
			ylabel={время, мкс},
			]
			\legend{ 
				Гномья сорт., 
				Сорт. перемешиванием, 
				Сорт. двоич. деревом
			};
			\addplot [
			solid, 
			draw = blue,
			mark = *, 
			mark options = {
				scale = 1.5, 
				fill = blue, 
				draw = black
			}
			] table [x={size}, y={time}] {gn1.csv};
			\addplot [
			dotted, 
			draw = red,
			mark = star, 
			mark options = {
				scale = 1.5, 
				draw = red
			}
			] table [x={size}, y={time}] {sh1.csv};
			\addplot [
			dashed, 
			draw = green,
			mark = x, 
			mark options = {
				scale = 1.5, 
				draw = green
			}
			] table [x={size}, y={time}] {tree1.csv};		
		\end{loglogaxis}
	\end{tikzpicture}
	\caption{Сравнение алгоритмов при случайных данных}
	\label{img:g1}
\end{figure}


\section*{Вывод} 
При заполнении массива случайными данными реализация алгоритма сортировки перемешиванием незначительно отличается по времени работы от реализации алгоритма гномьей сортировки: для всех размерностей разница не более 5~\%. Сортировка двоичным деревом  оказалась эффективнее других методов: при размерности 100 её реализация быстрее в 3 раза, при размерности 2500 быстрее в 57 раз.

