\chapter{Аналитическая часть}
В этом разделе представлено описание алгоритма.

\section{Алгоритм DBSCAN}
Алгоритм DBSCAN (Density Based Spatial Clustering of Applications with Noise), плотностный алгоритм для кластеризации пространственных данных с присутствием шума, был предложен Мартином Эстер, Гансом-Питером Кригель и коллегами в 1996 году как решение проблемы разбиения данных на кластеры произвольной формы [4].

Идея, положенная в основу алгоритма, заключается в том, что внутри каждого кластера наблюдается типичная плотность точек (объектов), которая заметно выше, чем плотность снаружи кластера, а также плотность в областях с шумом ниже плотности любого из кластеров. 
Ещё точнее, что для каждой точки кластера её соседство заданного радиуса должно содержать не менее некоторого числа точек, это число точек задаётся пороговым значением.
Перед изложением алгоритма дадим необходимые определения.

\textbf{Определение 1.} \emph{Eps-соседство точки p}, обозначаемое как $N_{eps}(p)$, определяется как множество документов, находящихся от точки $p$ на расстояния не более $Eps$.
Поиска точек, чьё $N_{eps}(p)$ содержит хотя бы минимальное число точек (\emph{MinPt}) не достаточно, так как точки бывают двух видов: ядровые и граничные.

\textbf{Определение 2.} Точка \emph{p непосредственно плотно-достижима} из точки $q$ (при
заданных \emph{Eps} и \emph{MinPt}), если $p \in N_{eps}(q)$ и $\vert N_{eps}(q) \vert \ge MinPt.$

\textbf{Определение 3.} Точка \emph{p плотно-достижима} из точки $q$ (при
заданных \emph{Eps} и \emph{MinPt}), если существует последовательность точек $p_1, ..., p_n$ такая, что при всех $i$ $p_{i+1}$ непосредственно плотно-достижима из $p_i.$

\textbf{Определение 4.} Точка \emph{p плотно-связана} с точкой $q$ (при
заданных \emph{Eps} и \emph{MinPt}), если существует точка $o$ такая, что $p$ и $q$ плотно-достижимы из неё.

\textbf{Определение 5.} \emph{Кластер} --- это не пустное множество плотно-связанных точек. В каждом кластере содержится хотя бы \emph{MinPt} объектов.

\textbf{Определение 6.} \emph{Шум} --- это множество точек, которые не принадлежат ни одному кластеру.

Алгоритм DBSCAN для заданных значений параметров Eps и MinPt исследует кластер следующим образом: сначала выбирает случайную точку, являющуюся ядровой, в качестве затравки, затем помещает в кластер саму затравку и все точки, плотно-достижимые из неё.

