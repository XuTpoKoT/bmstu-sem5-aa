\chapter{Конструкторская часть}

\section{Алгоритм DBSCAN}
На рисунке \ref{img:dbscan} приведена схема алгоритма DBSCAN, на рисунках \ref{img:expandCl}--\ref{img:treadCN} приведены схемы вспомогательных подпрограмм.

\imgHeight{170mm}{dbscan}{Схема алгоритма DBSCAN}
\clearpage

\imgHeight{170mm}{expandCl}{Функция ExpandCluster}
\clearpage

\imgHeight{100mm}{serialCN}{Функция последовательного вычисления соседей}

\imgHeight{100mm}{parCN}{Функция параллельного вычисления соседей: основной поток}

\imgHeight{170mm}{treadCN}{Функция параллельного вычисления соседей: вспомоготельные потоки}

\clearpage
\section{Функциональные требования}
К программе предъявляются следующие требования:
\begin{itemize}
	\item считывание из файла координат точек на плоскости;
	\item вывод в файл номеров кластеров точек;
	\item выполнение замеров времени выполнения реализаций алгоритмов.
\end{itemize}