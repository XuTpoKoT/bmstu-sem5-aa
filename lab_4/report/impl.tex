\chapter{Технологическая часть}

\section{Средства реализации}
Используемое программное обеспечение должно предоставлять возможность выполнения замеров времени и работы с потоками.

Для выполнения данной лабораторной работы был выбран язык программирования С++ \cite{c} и среда разработки CLion, которая позволяет замерять процессорное время с помощью пакета \texttt{<ctime>} \cite{ctime} и работать с потоками, используя класс std::thread \cite{thread}.

\section{Реализация алгоритмов}

В листингах \ref{lst:DBScan}--\ref{lst:getNeighbors} приведены класс DBScan, реализация алгоритма и вспомогательных функций.

\begin{lstlisting}[label=lst:DBScan,caption=Классс DBScan (часть 1)]
class DBScan {
	public:
	const vector<shared_ptr<Point>> points;
	vector<int> clusterIndexes;
	
	DBScan(vector<shared_ptr<Point>> points, size_t minPointsInCluster, double eps, int cntThreads) :
	points(points), minPointsInCluster(minPointsInCluster), eps(eps), threads(cntThreads) {
		this->cntThreads = cntThreads;
		
		for (size_t i = 0; i < points.size(); i++) {
			clusterIndexes.push_back(UNVISITED);
		}
	} 
	
	void runSerial();
	void runParallel();
\end{lstlisting}

\begin{lstlisting}[label=lst:DBScan2,caption=Классс DBScan (часть 2)]
private:
	const size_t minPointsInCluster;
	const double eps;
	int cntThreads;
	int curPointIndex;
	int curClusterID;
	vector<thread> threads;
	vector<vector<int>> pointGroups;
	map<int, vector<int>> neighbors;
	mutex m;
	void initNeighbors();
	void serialCalcNeighbors();
	void parallelCalcNeighbors();
	void formPointGroups();	
	int run();
	int expandCluster();
	vector<int> getNeighbors(const shared_ptr<Point> point);
	void calcGroupNeighbors(int groupNumber);
};
\end{lstlisting}

\begin{lstlisting}[label=lst:serialCalcNeighbors,caption=Функция последовательного вычисления соседей]
	void DBScan::serialCalcNeighbors() {
		for (int i = 0; i < points.size(); i++) {
			neighbors[i] = getNeighbors(points[i]);
		}
	}
\end{lstlisting}

\begin{lstlisting}[label=lst:parallelCalcNeighbors,caption=Функция параллельного вычисления соседей: основной поток]
	void DBScan::parallelCalcNeighbors() {
		formPointGroups();
		for (int i = 0; i < cntThreads; i++) {
			this->threads[i] = thread(&DBScan::calcGroupNeighbors, this, i);
		}
		for (int i = 0; i < cntThreads; i++) {
			this->threads[i].join();
		}
	}
	
\end{lstlisting}

\begin{lstlisting}[label=lst:calcGroupNeighbors,caption=Функция параллельного вычисления соседей: вспомогательные потоки]
void DBScan::calcGroupNeighbors(int groupNumber) {
	for (size_t i = 0, n = pointGroups[groupNumber].size(); i < n; i++) {
		vector<int> curNeighbors = getNeighbors(points[pointGroups[groupNumber][i]]);
		m.lock();
		neighbors[i] = move(curNeighbors);
		m.unlock();
	}
}
\end{lstlisting}

\begin{lstlisting}[label=lst:run,caption=Основная функция алгоритма]
int DBScan::run() {
	curClusterID = 0;
	for (size_t i = 0, n = points.size(); i < n; i++) {
		if (clusterIndexes[i] == UNVISITED) {
			curPointIndex = i;
			if (expandCluster() != -1) {
				curClusterID++;
			}
		}
	}
	
	return 0;
}
\end{lstlisting}

\begin{lstlisting}[label=lst:expandCluster,caption=Функция expandCluster (часть 1)]
int DBScan::expandCluster() {
	vector<int> seeds = getNeighbors(points.at(curPointIndex));
	if (seeds.size() + 1 < minPointsInCluster) {
		clusterIndexes[curPointIndex] = NOISE;
		return -1;
	}
	clusterIndexes[curPointIndex] = curClusterID;
	for (size_t i = 0, cntSeeds = seeds.size(); i < cntSeeds; ++i) {
		clusterIndexes[seeds[i]] = curClusterID;
	}
\end{lstlisting}

\begin{lstlisting}[label=lst:expandCluster2,caption=Функция expandCluster (часть 2)]
	while (seeds.size() > 0) {
		int curSeed = seeds.back();
		seeds.pop_back();
		vector<int> seedNeighbors = getNeighbors(points.at(curSeed));
		
		size_t cntNeighbors = seedNeighbors.size();
		if (cntNeighbors + 1 >= minPointsInCluster) {
			for (size_t i = 0; i < cntNeighbors; ++i) {
				int curNeighbor = seedNeighbors[i];
				if (clusterIndexes[curNeighbor] == UNVISITED || clusterIndexes[curNeighbor] == NOISE) {
					if (clusterIndexes[curNeighbor] == NOISE) {
						seeds.push_back(curNeighbor);
					}
					clusterIndexes[curNeighbor] = curClusterID;
				}
			}
		}
	}
	
	return 0;
}
\end{lstlisting}

\begin{lstlisting}[label=lst:getNeighbors,caption=Функция getNeighbors]
vector<int> DBScan::getNeighbors(const shared_ptr<Point> point) {
	vector<int> neighborIndexes;
	for (size_t i = 0, n = points.size(); i < n; i++) {
		double distance = point->dist(points[i]);
		if (distance <= eps && distance >= 1e-8) {
			neighborIndexes.push_back(i);
		}
	}
	
	return neighborIndexes;
}
\end{lstlisting}

\section{Пример работы программы}
На рисунках \ref{img:0cl}--\ref{img:2cl} приведены результаты работы программы при различных значениях \emph{Eps} и \emph{MinPt}. 

\imgHeight{130mm}{0cl}{Все точки являются шумом}

\imgHeight{100mm}{1cl}{1 кластер}

\imgHeight{100mm}{2cl}{2 кластера}




