\chapter{Технологическая часть}

В данном разделе представлены требования к программному обеспечению, средства реализации, листинги кода и проведена оценка трудоёмкости.

\section{Требования к программному \newline обеспечению}

Используемое программное обеспечение должно предоставлять возможность измерения процессорного времени.

\section{Средства реализации}

Для реализации данной лабораторной работы был выбран язык программирования С++ \cite{c} и среда разработки CLion, которая позволяет замерять процессорное время с помощью пакета \texttt{<ctime>} \cite{ctime}.

\section{Реализация алгоритмов}

В листингах \ref{lst:shakerSort}--\ref{lst:treeSort} приведены реализации алгоритмов.

\clearpage
\begin{lstlisting}[label=lst:gnomeSort,caption=Функция гномьей сортировки]
void gnomeSort(std::vector<int> &a) {
	int n =  a.size();
	
	for (int i = 0; i < n;) {
		if (i == 0 || a[i-1] <= a[i]) {
			i++;
		} else {
			int tmp = a[i];
			a[i] = a[i-1];
			a[i-1] = tmp;
			i--;
		}
	}
}	
\end{lstlisting}
