\chapter{Технологическая часть}

\section{Средства реализации}

В качестве языка программирования для реализации данной лабораторной работы был выбран ЯП Python \cite{py}. В качестве среды разработки выбор сделан в сторону Visual Studio Code.

\section{Реализация алгоритма}

В листинге \ref{lst:l1} приведена реализация функции чтения запроса.

\begin{lstlisting}[label=lst:l1,caption=Функция чтения запроса]
def readRequest():
	reqIsValid = False
	sizeBordersArr = []
	reqWords = input("Input request: ").lower().split()
	for word in reqWords:
		if damLev(word, "crossovki") <= 2:
			break
	else:
		return reqIsValid, sizeBordersArr
	
	for word in reqWords:
		for attr in attributes:
			if damLev(word, attr) <= 3:
				sizeBordersArr.append(categories[attr])
	
	if len(sizeBordersArr) > 0:
		reqIsValid = True
	
	return reqIsValid, sizeBordersArr
\end{lstlisting}



